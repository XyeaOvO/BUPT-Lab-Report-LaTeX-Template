\section{示例文档导览}\label{sec:overview}
\begin{tipbox}
本示例以“开关式模块化”为核心:在 `tex/meta.tex` 中集中控制算法、代码高亮、参考文献、术语表、索引、批注与水印等功能,主文档保持干净稳定。
\end{tipbox}

\begin{warnbox}
若启用 `minted`,编译需加入 `-shell-escape`;使用 `biblatex` 需要 biber;术语表与缩略语需运行 `makeglossaries`。详情可参考 \url{https://ctan.org/pkg/minted} 与 \url{https://ctan.org/pkg/glossaries}。
\end{warnbox}

\begin{alertbox}
水印用于草稿流转,最终提交前请将 `\UseWatermarkfalse` 关闭,避免正式文档出现“DRAFT”。
\end{alertbox}

本示例围绕数字信号处理场景组织内容,通过第\ref{sec:math}节、第\ref{sec:algorithm}节、第\ref{sec:tables}节、第\ref{sec:circuit}节与第\ref{sec:plots}节展示数学、算法、表格、图像、电路与绘图模块的协同工作,并以交叉引用确保一致性。示例中用 \SI{48}{\kilo\hertz} 和 \SI{3.3}{\kilo\hertz} 标注参数,展示 `siunitx` 的规范单位写法。

\ifUseTodo
\todo[inline]{将“实验结果分析”替换为你的实际项目结论,并添加关键图表。}
\fi

\ifUseBiblatex
模板示例引用了经典文献 \cite{knuth1984texbook,oppenheim2010dsp,cooley1965algorithm},展示自动编号与参考文献列表。
\fi
\ifUseGlossaries
术语示例:\gls{dft}、\gls{fft}、\gls{nyquist}、\gls{psd};缩略语示例:\gls{adc}、\gls{snr}、\gls{lti}。
\fi
\ifUseIndex
关键术语索引:数字信号处理\index{数字信号处理}、频谱分析\index{频谱分析}、窗函数\index{窗函数}、采样定理\index{采样定理}。
\fi

\subsection{编译与流程}
\begin{enumerate}[label=\textcolor{BUPTBlue}{\bfseries \arabic*.}]
  \item 先运行一次编译生成辅助文件,再执行 biber 与 makeglossaries,最后二次编译更新引用与索引。
  \item 图表与公式统一用 \verb|\cref| 引用,避免手动编号出错,见 \cref{eq:dft,alg:os,tab:spec,fig:layout,fig:plot}。
  \item 若需要快速关闭某模块,仅需切换 `tex/meta.tex` 中的布尔开关。
\end{enumerate}
