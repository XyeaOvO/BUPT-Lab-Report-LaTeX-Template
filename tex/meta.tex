% =========================================================
% 0) 元信息:只改这里
% =========================================================
\newcommand{\SchoolName}{Future School, BUPT}
\newcommand{\CourseName}{现代信号处理}
\newcommand{\ReportType}{综合示范文档}
\newcommand{\ReportTitle}{LaTeX实验报告模板}
\newcommand{\ExperimentName}{示范项目:模板工程化与全功能演示}
\newcommand{\AuthorName}{王嘉然}
\newcommand{\StudentID}{2023212666}
\newcommand{\ClassName}{23级}
\newcommand{\Instructor}{教授}
\newcommand{\LogoFile}{images/logo.png} % 不存在则自动绘制简易Logo
\newcommand{\ReportDate}{\today}

% =========================================================
% 1) 功能开关(true/false)
% =========================================================
% === 字体与数学公式 ===
\newif\ifUseUnicodeMath \UseUnicodeMathtrue   % 开关:是否使用 unicode-math 宏包
                                              % 注释:仅适用于 XeLaTeX 或 LuaLaTeX 引擎。
                                              %       利用 Unicode 字体排版数学公式。
                                              %       如果使用 pdfLaTeX 或与其他数学包冲突,请设为 false。

% === 代码高亮 ===
\newif\ifUseMinted      \UseMintedtrue        % 开关:是否使用 minted 宏包进行代码高亮
                                              % 注释:minted 效果很好但依赖 Python 的 Pygments 库。
                                              %       注意:启用此项编译时必须添加命令行参数 -shell-escape。

% === 绘图工具 ===
\newif\ifUseCircuitikz  \UseCircuitikztrue    % 开关:是否使用 circuitikz 宏包
                                              % 注释:用于绘制电路图(基于 TikZ)。

\newif\ifUsePgfplots    \UsePgfplotstrue      % 开关:是否使用 pgfplots 宏包
                                              % 注释:用于绘制高质量的科学函数图、数据统计图(2D/3D)。

% === 文献与索引 ===
\newif\ifUseBiblatex    \UseBiblatextrue      % 开关:是否使用 biblatex 进行文献管理
                                              % 注释:现代的文献管理方式,替代传统的 BibTeX。

\newif\ifUseGlossaries  \UseGlossariestrue    % 开关:是否使用 glossaries 宏包
                                              % 注释:用于制作术语表、缩略语表(Acronyms)。

\newif\ifUseIndex       \UseIndextrue         % 开关:是否生成索引
                                              % 注释:通常对应 makeidx 或 imakeidx 宏包,用于生成文末的书籍索引。

% === 辅助工具 ===
\newif\ifUseTodo        \UseTodofalse          % 开关:是否使用 todonotes 宏包
                                              % 注释:用于在文档边缘或文中添加“待办事项”标注,定稿时通常设为 false。

\newif\ifUseWatermark   \UseWatermarkfalse     % 开关:是否添加水印
                                              % 注释:用于在页面背景添加“草稿”、“机密”或校徽等水印。

% === 页面状态逻辑 ===
% Page-count state: set true for main matter
\newif\ifMainMatter
\MainMatterfalse                              % 开关:标记当前是否处于“正文(Main Matter)”部分
                                              % 注释:这不是加载宏包的开关,而是一个逻辑状态标记。
                                              %       通常在书籍类文档中,前言(FrontMatter)用罗马数字页码,
                                              %       进入正文(MainMatter)后改为阿拉伯数字。
                                              %       这个开关可能用于控制页眉页脚的显示格式或页码计数逻辑。
