% =========================================================
% 2) 引擎与字体(修正版:修复空格bug,增加Windows保底)
% =========================================================
\usepackage{iftex}

\ifPDFTeX
  \PackageError{lab-report}{Please use XeLaTeX or LuaLaTeX}{}
\else
  \usepackage{fontspec}
  
  % --- 字体设置辅助宏 (使用 expl3 语法,安全处理空格) ---
  \ExplSyntaxOn
  % 定义一个命令:\SetFontWithFallback{设置命令}{选项}{字体列表}
  \NewDocumentCommand{\SetFontWithFallback}{ m m m }
    {
      \bool_set_false:N \l_tmpa_bool
      \clist_map_inline:nn { #3 }
        {
          \fontspec_font_if_exist:nTF { ##1 }
            {
              #1 [ #2 ] { ##1 }
              \bool_set_true:N \l_tmpa_bool
              \clist_map_break: % 找到第一个存在的字体后立即停止
            }
            {}
        }
      % 如果遍历完都没找到,报个警告
      \bool_if:NF \l_tmpa_bool
        {
          \msg_warning:nnn { font-setup } { font-not-found } { #3 }
        }
    }
  \ExplSyntaxOff

  % --- 1. 西文字体配置 ---
  % Main: 优先 Pagella (Palatino), 其次 Times New Roman
  \SetFontWithFallback{\setmainfont}{
    Ligatures=TeX
  }{TeX Gyre Pagella, Palatino Linotype, Times New Roman}
  
  % Sans: 优先 Heros (Helvetica), 其次 Arial
  \SetFontWithFallback{\setsansfont}{
    Scale=MatchLowercase
  }{TeX Gyre Heros, Helvetica, Arial}
  
  % Mono: 优先 JetBrains Mono, 其次 Consolas
  \SetFontWithFallback{\setmonofont}{
    Scale=MatchLowercase
  }{JetBrains Mono, Consolas, Courier New}

  % --- 2. 中文字体配置 ---
  % 注意:ctexart 会自动加载 xeCJK,这里我们覆盖它的设置
  % 优先:思源宋体/黑体;保底:Windows 中易宋体/黑体 (SimSun/SimHei)
  
  \ifXeTeX
    % XeTeX 下开启伪粗体 (AutoFakeBold),防止粗体不显示
    \SetFontWithFallback{\setCJKmainfont}{
      AutoFakeBold=2, AutoFakeSlant=0.15
    }{思源宋体 CN, Source Han Serif SC, Noto Serif CJK SC, SimSun, 宋体}
    
    \SetFontWithFallback{\setCJKsansfont}{
      AutoFakeBold=2
    }{思源黑体, Source Han Sans SC, Noto Sans CJK SC, Microsoft YaHei, 微软雅黑, SimHei, 黑体}
  \else
    % LuaLaTeX 配置
    \SetFontWithFallback{\setCJKmainfont}{
    }{思源宋体 CN, Source Han Serif SC, Noto Serif CJK SC, SimSun, 宋体}
    
    \SetFontWithFallback{\setCJKsansfont}{
    }{思源黑体, Source Han Sans SC, Noto Sans CJK SC, Microsoft YaHei, 微软雅黑, SimHei, 黑体}
  \fi

  % --- 3. 数学字体 ---
  \usepackage{amsmath}
  \usepackage{mathtools} 
  
  \ifUseUnicodeMath
    % --- A. 现代模式:Unicode Math ---
    \usepackage[math-style=ISO,bold-style=ISO]{unicode-math}
    
    % 配置数学字体 (使用 fontspec 原生命令检测即可)
    \IfFontExistsTF{TeX Gyre Pagella Math}{
      \setmathfont{TeX Gyre Pagella Math}
    }{
      \setmathfont{Latin Modern Math}
    }
    
    % 兼容性定义:让 \bm{} 在 unicode-math 下也能用 (映射为 \symbf)
    \providecommand{\bm}[1]{\symbf{#1}}
  \else
    % --- B. 传统模式:PDFLaTeX/XeLaTeX 传统数学 ---
    \usepackage{amssymb} 
    \usepackage{bm} 
  \fi
\fi

% =========================================================
% 3) 版式与基础包
% =========================================================
\usepackage[a4paper,top=2.4cm,bottom=2.4cm,left=2.2cm,right=2.2cm]{geometry}
\usepackage{microtype}
\usepackage{setspace}
\setstretch{1.15}

\setlength{\parindent}{2em}
\setlength{\parskip}{0.25em}

\usepackage{graphicx}
\graphicspath{{figs/}{images/}{./}}
\usepackage{float}
\usepackage{placeins}
\usepackage{booktabs,threeparttable,tabularx}
\usepackage[table]{xcolor}
\usepackage{xurl}
\usepackage{pdfpages}

% 单位
\usepackage{siunitx}
\sisetup{
  per-mode=symbol,
  separate-uncertainty=true,
  detect-all=true
}

% 列表
\usepackage{enumitem}
\setlist[itemize]{leftmargin=2em,itemsep=0.2em,topsep=0.3em}
\setlist[enumerate]{leftmargin=2em,itemsep=0.2em,topsep=0.3em}

% 图表标题
\usepackage{caption}
\usepackage{subcaption}
\captionsetup{font=small,labelfont=bf,labelsep=quad}

% =========================================================
% 4) 颜色与超链接(集中配置)
% =========================================================
\definecolor{BUPTBlue}{RGB}{0,61,124}
\definecolor{BUPTLight}{RGB}{66,139,202}
\definecolor{GrayText}{RGB}{110,110,110}
\definecolor{CodeBg}{RGB}{248,248,250}
\definecolor{AlertRed}{RGB}{220,53,69}
\definecolor{InfoGreen}{RGB}{40,167,69}
\definecolor{WarnOrange}{RGB}{255,159,10}

\usepackage{hyperref}
\usepackage{bookmark}
\usepackage[nameinlink,noabbrev]{cleveref}

\hypersetup{
  unicode=true,
  psdextra=true,
  colorlinks=true,
  linkcolor=BUPTBlue,
  urlcolor=BUPTLight,
  citecolor=BUPTBlue,
  pdftitle={\ReportType: \ReportTitle},
  pdfauthor={\AuthorName}
}

% 避免书签里出现命令警告
\pdfstringdefDisableCommands{%
  \def\LaTeX{LaTeX}%
}


\crefname{figure}{图}{图}
\crefname{table}{表}{表}
\crefname{equation}{式}{式}
\crefname{algorithm}{算法}{算法}
\crefname{algocf}{算法}{算法}

\numberwithin{equation}{section}
\numberwithin{figure}{section}
\numberwithin{table}{section}

% =========================================================
% 5) 标题样式:统一的左侧竖条风格
% =========================================================
\ctexset{
  % --- 一级标题 ---
  section = {
    format = \Large\bfseries\color{BUPTBlue},
    % 竖条样式:3pt宽,右侧留空 0.5em
    name = {\vrule width 3pt\hspace{0.5em}},
    number = \thesection,
    aftername = \hspace{0.5em},
    beforeskip = 1.5ex plus .2ex,
    afterskip = 1.0ex plus .2ex,
  },
  % --- 二级标题 ---
  subsection = {
    format = \large\bfseries\color{black},
    % 竖条样式:2pt宽,深灰色
    name = {\color{gray!70}\vrule width 2pt\hspace{0.5em}},
    number = \thesubsection,
    aftername = \hspace{0.5em},
    beforeskip = 1.2ex plus .2ex,
    afterskip = 0.6ex plus .2ex
  },
  % --- 三级标题 ---
  subsubsection = {
    format = \normalsize\bfseries\color{black},
    % 竖条样式:1pt宽,浅灰色
    name = {\color{gray!40}\vrule width 1pt\hspace{0.5em}},
    number = \thesubsubsection,
    aftername = \hspace{0.5em},
    beforeskip = 1.0ex plus .2ex,
    afterskip = 0.5ex plus .2ex
  }
}

% =========================================================
% 6) 页眉页脚
% =========================================================
\usepackage{fancyhdr}
\usepackage{lastpage}

% 防止 fancyhdr 报 "headheight is too small" 警告
\setlength{\headheight}{28pt}
\savegeometry{main} 

% Logo 绘制逻辑:仅当文件存在时显示,移除之前的文字替代方案
\newcommand{\DrawLogo}{%
  \IfFileExists{\LogoFile}{%
    \includegraphics[height=0.8cm]{\LogoFile}%
  }{}% 如果文件不存在,什么也不做,保持静默
}

\pagestyle{fancy}
\fancyhf{}

% --- 页眉设置 ---
% \fancyhead[L]{\DrawLogo} % 根据需要取消注释
\fancyhead[R]{\small\sffamily\color{GrayText}\CourseName\ \textbar\ \AuthorName}

% --- 页眉分割线(使用 leaders 绘制精准颜色的实线) ---
\renewcommand{\headrulewidth}{0.6pt}
\renewcommand{\headrule}{%
  \hbox to\headwidth{\color{BUPTBlue}\leaders\hrule height \headrulewidth\hfill}%
}

% --- 页脚设置 ---
\fancyfoot[L]{\small\sffamily\color{GrayText}\ReportTitle}

% [重点改进] 中间页脚:使用 makebox 固定宽度,防止页码位数变化导致色块抖动
\fancyfoot[C]{%
  \raisebox{0pt}[0pt][0pt]{%
    \colorbox{BUPTBlue}{%
      \makebox[2em][c]{% 设定 2em 固定宽度,内容居中
        \color{white}\sffamily\bfseries\strut\thepage
      }%
    }%
  }%
}

\fancyfoot[R]{\small\sffamily\color{GrayText}\ifMainMatter 共\pageref{LastPage} 页\fi}

% =========================================================
% 7) TikZ / 电路图 / 绘图(按开关加载)
% =========================================================
\usepackage{tikz}
\usetikzlibrary{calc,positioning,arrows.meta,shapes.geometric}

\ifUseCircuitikz
  \usepackage[europeanresistors,americaninductors]{circuitikz}
  \ctikzset{bipoles/length=1.0cm}
\fi

\ifUsePgfplots
  \usepackage{pgfplots}
  \pgfplotsset{compat=1.18}
\fi

% =========================================================
% 8) 算法(algorithm2e)
% =========================================================
\usepackage[ruled,vlined,linesnumbered]{algorithm2e}
\renewcommand{\algorithmcfname}{算法}
\SetKwComment{Comment}{$\triangleright$\ }{}
\DontPrintSemicolon

% =========================================================
% 9) 盒子/定理系统(集中定义样式,避免重复)
% =========================================================
\usepackage[most]{tcolorbox}
\tcbuselibrary{skins,breakable,theorems,listings}

\tcbset{
  boxbase/.style={
    enhanced, breakable,
    boxrule=0pt, leftrule=3mm,
    arc=2mm,
    before skip=0.6em,
    after skip=0.6em
  },
  theobase/.style={
    enhanced, breakable,
    arc=3mm,
    fonttitle=\bfseries,
    attach boxed title to top left={yshift=-2mm,xshift=5mm},
    boxed title style={boxrule=0pt}
  }
}

\newtcbtheorem[auto counter,number within=section]{definition}{定义}{
  theobase,
  colback=BUPTBlue!4, colframe=BUPTBlue,
  coltitle=white,
  boxed title style={colback=BUPTBlue,colframe=BUPTBlue},
  separator sign={\ $\blacktriangleright$}
}{def}

\newtcbtheorem[auto counter,number within=section]{theorem}{定理}{
  theobase,
  colback=WarnOrange!7, colframe=WarnOrange!70!black,
  coltitle=white,
  boxed title style={colback=WarnOrange!75!black,colframe=WarnOrange!75!black},
  separator sign={\ $\mid$}
}{thm}

\newtcolorbox{tipbox}[1][]{boxbase,colback=InfoGreen!8,colframe=InfoGreen!70!black,title={提示},fonttitle=\bfseries,#1}
\newtcolorbox{warnbox}[1][]{boxbase,colback=WarnOrange!10,colframe=WarnOrange!80!black,title={注意},fonttitle=\bfseries,#1}
\newtcolorbox{alertbox}[1][]{boxbase,colback=AlertRed!10,colframe=AlertRed,title={警告},fonttitle=\bfseries,#1}

% 简易证明环境(不引 amsthm,避免定理系统打架)
\newcommand{\qedzh}{%
  \tikz[baseline=-0.6ex, line width=0.5pt]\draw(0,0) rectangle (1.3ex,1.3ex);%
}
\newenvironment{proofzh}{\par\noindent\textbf{证明:}}{\hfill\qedzh\par}

% =========================================================
% 10) 代码环境:Mac 窗口风格 (修复行号溢出与美观度)
% =========================================================

% --- 1. 颜色定义 ---
\definecolor{macRed}{RGB}{255,95,86}
\definecolor{macYellow}{RGB}{255,189,46}
\definecolor{macGreen}{RGB}{39,201,63}
\definecolor{macBar}{RGB}{235,235,235}  % 标题栏浅灰色
\definecolor{macBorder}{RGB}{200,200,200} % 边框颜色

% --- 2. 绘制 Mac 三色圆点 ---
\newcommand{\MacDots}{%
  \tikz[baseline=-0.3ex, yscale=1, xscale=1]{
    \fill[macRed] (0,0) circle (2.5pt);
    \fill[macYellow] (0.6,0) circle (2.5pt);
    \fill[macGreen] (1.2,0) circle (2.5pt);
  }%
}

% --- 3. 字体设置 (可选,推荐用更好看的等宽字体) ---
% 如果电脑没装这些字体,LaTeX 会使用默认的 ttfamily,不会报错
\IfFontExistsTF{JetBrains Mono}{
    \setmonofont[Scale=MatchLowercase]{JetBrains Mono}
}{
    \IfFontExistsTF{Consolas}{
        \setmonofont[Scale=MatchLowercase]{Consolas}
    }{}
}

\ifUseMinted
  % ==================== Minted 引擎 ====================
  \usepackage{minted}
  \tcbuselibrary{minted}

  % 修改为 [3][] : 接受三个参数 (#1=可选配置, #2=文件名/标题, #3=语言)
  \newtcblisting{maccode}[3][]{
    enhanced, breakable,
    colback=white, colframe=macBorder, colbacktitle=macBar, coltitle=black,
    arc=5pt, boxrule=0.5pt,
    left=1mm, right=1mm, top=1mm, bottom=1mm,
    % 使用 #2 作为标题 (文件名)
    title={\MacDots\quad \sffamily\small\color{gray!60}\detokenize{#2}},
    toptitle=2pt, bottomtitle=2pt,
    listing only,
    listing engine=minted,
    % 使用 #3 作为语言
    minted language=#3,
    minted options={
        fontsize=\small, breaklines=true, linenos=true, autogobble, 
        tabsize=4, numbersep=5pt, xleftmargin=10pt
    },
    #1 % 允许插入额外选项
  }

\else
  % ==================== Listings 引擎 (默认推荐) ====================
  \usepackage{listings}

  % 定义 listings 样式
  \lstdefinestyle{MacStyle}{
    basicstyle=\ttfamily\small, % 字号
    keywordstyle=\color{BUPTBlue}\bfseries,
    commentstyle=\color{gray!60}\itshape,
    stringstyle=\color{macGreen!80!black},
    numbers=left,
    numberstyle=\tiny\color{gray!50}\sffamily, % 行号样式:浅灰、无衬线
    numbersep=8pt,   % 行号和代码的距离
    
    % [关键] 布局修正
    xleftmargin=2.5em,  % 整体左缩进:给行号留出足够空间!
    framexleftmargin=0em,
    
    breaklines=true,
    tabsize=4,
    showstringspaces=false,
    captionpos=b,
    
    % 紧凑布局
    aboveskip=0pt,
    belowskip=0pt,
    columns=flexible, % 字符间距自然,不僵硬
    keepspaces=true
  }

  \newtcblisting{maccode}[3][]{
    enhanced, breakable,
    colback=white, colframe=macBorder, colbacktitle=macBar, coltitle=black,
    arc=6pt, boxrule=0.5pt,
    % 使用 #2 作为标题 (文件名)
    title={\MacDots\quad \sffamily\small\color{gray!60}\detokenize{#2}},
    toptitle=3pt, bottomtitle=3pt,
    listing only,
    % 使用 #3 作为语言
    listing options={style=MacStyle, language=#3},
    left=0pt, right=0pt, top=2pt, bottom=2pt,
    #1
  }
\fi