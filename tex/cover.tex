\newcommand{\MakeCover}{%
  \begin{titlepage}
    \thispagestyle{empty}
    \newgeometry{top=3.0cm,bottom=2.5cm,left=2.5cm,right=2.5cm}
    \begingroup
    \setlength{\parindent}{0pt}
    
    % --- 1. 背景层 (TikZ) ---
    \begin{tikzpicture}[remember picture,overlay]
      % 左上角信号源:增加一点模糊感(可选),这里保持清晰几何
      \fill[BUPTBlue, opacity=0.06] (current page.north west) circle (14cm);
      
      % 右下角装饰:使用 scope 裁切,确保不论画多大圆都不会超出页面报错(理论上 overlay 不会,但为了保险)
      \begin{scope}
        \clip (current page.south west) rectangle (current page.north east);
        \fill[BUPTBlue, opacity=0.9] 
          ($(current page.south east)+(-1.5cm, 5cm)$) circle (6cm);
        \fill[BUPTBlue!60, opacity=0.7] 
          ($(current page.south east)+(-5cm, 2.5cm)$) circle (3.5cm);
      \end{scope}
      
      % 顶部装饰线:精确控制位置
      \draw[BUPTBlue, line width=2.5pt] 
        ($(current page.north west)+(2.5cm,-2.5cm)$) -- 
        ($(current page.north east)+(-2.5cm,-2.5cm)$);
    \end{tikzpicture}

    % --- 2. 顶部 Header 区 ---
    \vspace*{1cm} % 避开装饰线
    \noindent
    \begin{minipage}[b]{0.7\textwidth}
      {\fontsize{18}{22}\selectfont\bfseries\color{BUPTBlue}\SchoolName}\\[0.5em]
      {\large\color{gray}\sffamily\CourseName\ \textbar\ \ReportType}
    \end{minipage}%
    \hfill
    \begin{minipage}[b]{0.25\textwidth}
      \raggedleft
      \resizebox{!}{2cm}{\DrawLogo}
    \end{minipage}

    % --- 3. 核心标题区 (使用弹性间距) ---
    \vspace{3cm} % 固定最小间距
    \vspace{0pt plus 1fill}

    \noindent
    {\Large\sffamily\color{BUPTBlue}\ExperimentName}\\[0.8cm]
    
    {
      \linespread{1.3}\selectfont 
      \fontsize{32}{38}\selectfont\bfseries\color{black}%
      \parbox[t]{\textwidth}{\raggedright \ReportTitle}
    }\\[0.8cm]
    
    {\color{BUPTBlue}\rule{3.5cm}{3.5pt}}\par

    % --- 4. 底部信息区 ---
    \vspace{0pt plus 2fill} % [优化3] 下方弹性力度是上方的2倍,使标题视觉重心偏上

    \noindent
    \begin{minipage}{0.7\textwidth}
      \renewcommand{\arraystretch}{1.6} % 表格行高
      \sffamily\large
      \newcommand{\InfoLabel}[1]{\makebox[4em][s]{\textbf{##1}}} 
      
      \begin{tabular}{@{}l@{\hspace{1em}}l@{}}
         \color{gray}\InfoLabel{学生姓名} & \bfseries \AuthorName \\
         \color{gray}\InfoLabel{学\hfill 号}   & \bfseries \StudentID \\ % 使用hfill自动撑开
         \color{gray}\InfoLabel{专业班级} & \bfseries \ClassName \\
         \color{gray}\InfoLabel{指导教师} & \bfseries \Instructor \\
         \color{gray}\InfoLabel{提交日期} & \bfseries \ReportDate
      \end{tabular}
    \end{minipage}

    % 底部留白,防止文字贴底
    \vspace*{1.5cm}
    
    \endgroup
  \end{titlepage}
  \loadgeometry{main}
}
