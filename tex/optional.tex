% =========================================================
% 12) 可选:参考文献 / 术语表 / 索引 / 批注 / 水印
% =========================================================
\ifUseBiblatex
  \usepackage{csquotes}
  \usepackage[
    backend=biber,
    style=ieee,       % 直接使用 ieee 风格,比 generic numeric 更标准
    sorting=none,     % 按照文中出现的顺序排序(这是 CS 的规矩)
    doi=false,        % 许多时候不需要显示 DOI,除非学校要求
    isbn=false,       % 不需要显示 ISBN
    url=false         % 除非是网页引用,否则通常不显示论文 URL
  ]{biblatex}
  \addbibresource{refs.bib}
\fi

\ifUseGlossaries
  \usepackage[acronym,toc]{glossaries}
  \makeglossaries
  \newglossaryentry{dft}{
  name={DFT},
  description={离散傅里叶变换}
}

\newglossaryentry{fft}{
  name={FFT},
  description={快速傅里叶变换}
}

\newglossaryentry{nyquist}{
  name={奈奎斯特采样定理},
  description={采样频率不低于信号最高频率两倍时可无失真重建}
}

\newglossaryentry{fir}{
  name={FIR},
  description={有限长单位冲激响应滤波器}
}

\newglossaryentry{iir}{
  name={IIR},
  description={无限长单位冲激响应滤波器}
}

\newglossaryentry{psd}{
  name={PSD},
  description={功率谱密度}
}

\newglossaryentry{quantization}{
  name={量化},
  description={将连续幅值映射为离散电平码}
}

\newacronym{adc}{ADC}{模数转换器}
\newacronym{snr}{SNR}{信噪比}
\newacronym{lti}{LTI}{线性时不变系统}

\fi

\newif\ifUseIndex \UseIndextrue % 确保定义了开关

\ifUseIndex
  \usepackage{imakeidx}

  \begin{filecontents*}{mystyle.ist}
    headings_flag 1
    heading_prefix "{\\bfseries\\Large "
    heading_suffix "}\\nopagebreak"  % 删掉了这里的 \n,防止报错
    symhead_positive "符号"
    numhead_positive "数字"
    delim_0 "\\dotfill "
  \end{filecontents*}

  % 保持 noautomatic
  \makeindex[intoc, options=-s mystyle.ist, title=索引, noautomatic]
\fi

\ifUseTodo
  \usepackage[colorinlistoftodos]{todonotes}
\else
  % 如果开关关闭,我们需要定义空的命令
  % 否则编译时遇到 \todo 会报错说 "Undefined control sequence"
  \usepackage[disable]{todonotes}
\fi

\ifUseWatermark
  \usepackage{xcolor}
  \usepackage{background}
  \backgroundsetup{
    scale=1,
    color=gray,
    opacity=0.10,
    angle=45,
    position={current page.center},
    contents={{\heiti\bfseries\fontsize{60}{60}\selectfont 草稿}}
  }
\fi
